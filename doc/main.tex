\documentclass[a4paper,12pt]{article}

\usepackage{cmap}
\usepackage{amsmath}
\usepackage[T2A]{fontenc}
\usepackage[utf8]{inputenc}
\usepackage[english,russian]{babel}


\begin{document}
\section{Модель движения}

Положение робота описывается в некоторой локальной системе отсчета $I$, ось $Z_I$ направлена против гравитации, оси $X_I$ и $Y_I$ дополняют систему до правой тройки.

Ориентация робота представлена кватернионом $q$ таким образом, что произвольный вектор, записанный в собственных осях $B$ робота проецируется в локальную систему как
$r_I = q \circ r_B \circ \tilde{q}$

Движение робота определено текущей ориентацией и управляющим воздействием на приводы колес $v$.
\begin{align*} 
&\dot{r}_B = (v \quad 0 \quad 0)^T \\
&\dot{r}_I = q \circ \dot{r}_B \circ \tilde{q}
\end{align*}

Текущая ориентация робота определяется плоской кривизной траектории робота $u$, которая регулируется поворотом рулей, и вектором нормали $n_s$ к поверхности, по которой перемещается робот.
\begin{align*} 
&q = q_{\phi} \circ q_{s} \\
&q_{\phi} = (\cos{\frac{\phi}{2}} \quad n_{s}\sin{\frac{\phi}{2}})^T \\
&\dot{\phi} = vu \\
&q_{s} = (\cos{\frac{\psi}{2}} \quad n_{\perp}\sin{\frac{\psi}{2}})^T \\
&\cos{\psi} = e_z \cdot n_s \\
& n_{\perp} ||n_{\perp}|| = e_z \times n_s 
\end{align*}
Поверхность может быть задана как $f_s(x,y,z) = 0$,
тогда 
$$n_s = \frac{\nabla f_s}{||\nabla f_s||}.$$

\section{Модель измерений}
	




\end{document}