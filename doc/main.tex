\documentclass[a4paper,12pt]{article}

\usepackage{cmap}
\usepackage{amsmath}
\usepackage[T2A]{fontenc}
\usepackage[utf8]{inputenc}
\usepackage[english,russian]{babel}

\begin{document}
\section{Модель движения}

Положение робота описывается в некоторой локальной системе отсчета $I$, ось $Z_I$ направлена против гравитации, оси $X_I$ и $Y_I$ дополняют систему до правой тройки.

Ориентация робота представлена кватернионом $q$ таким образом, что произвольный вектор, записанный в собственных осях $B$ робота проецируется в локальную систему как
$r_I = q_{IB} \circ r_B \circ \tilde{q}_{IB}$

Движение робота определено текущей ориентацией и управляющим воздействием на приводы колес $v$.
\begin{align} \label{eq:model_velocity}
&\dot{r}_B = (v \quad 0 \quad 0)^T \\
&\dot{r}_I = q_{IB} \circ \dot{r}_B \circ \tilde{q}_{IB}
\end{align}

Текущая ориентация робота определяется поворотом руля, которая определяет локальную кривизну траектории $u$, и вектором нормали $n_s$ к поверхности, по которой перемещается робот.
\begin{align} 
&q_{IB} = q_{\phi} \circ q_{s} \\
&q_{\phi} = (\cos{\frac{\phi}{2}} \quad n_{s}\sin{\frac{\phi}{2}})^T \\
&\dot{\phi} = vu \\
&q_{s} = (\cos{\frac{\psi}{2}} \quad n_{\perp}\sin{\frac{\psi}{2}})^T \\
&\cos{\psi} = e_z \cdot n_s \\
& n_{\perp} ||n_{\perp}|| = e_z \times n_s 
\end{align}

Угловая скорость связана с кватернионом ориентации уравнением Пуассона
\begin{align}  \label{eq:model_rot_velocity}
&\dot{q}_{IB} = \frac{1}{2} q_{IB} \circ \Omega_{B}
\end{align}
Поверхность может быть задана как $f_s(x,y,z) = 0$,
тогда 
$$n_s = \frac{\nabla f_s}{||\nabla f_s||}.$$


\section{Модель измерений}
	
\subsection{Датчик линейного ускорения}
Обозначим $\delta r^{imu}$ смещение акселерометра,  $q_{BM}$ -- ориентацию собственных осей акселерометра относительно осей $B$, тогда для произвольного вектора
$r_B = q_{BM} \circ r_M \circ \tilde{q}_{BM}.$

Тогда, показания акселерометра $a^{imu}$ складываются из переносной, вращательной и центробежной компонент, ускорения свободного падения и ошибки измерений $e^{imu}$.
\begin{align} 
\begin{split} \label{eq:imu_mes}
&q_{BM} \circ (a^{imu} - e^{imu}) \circ \tilde{q}_{BM} = \\
&\quad \quad \quad {q}_{BI} \circ (\ddot{r}_{I} - g_I) \circ \tilde{q}_{BI}
 - {\delta r^{imu}_B}  \times \dot{\Omega}_B
 - \Omega_B \times (\delta r^{imu}_B \times \Omega_B)
\end{split}
\\
&q_{BI} = \tilde{q}_{IB}
\end{align}

Будем считать, что ошибка измерений акселерометра складывается из постоянной компоненты  и белого шума
\begin{align} 
e^{imu} = e_{0}^{imu} + \epsilon^{imu}
\end{align}

Смещение и поворот собственных осей, а также постоянную компоненту ошибки измерений $e_{0}^{imu}$ можно найти с помощью калибровки.

\subsection{Датчик угловой скорости}
Обозначим $\delta r^{\textit{gyr}}$ смещение гироскопа,  $q_{BG}$ -- ориентацию собственных осей гироскопа относительно осей $B$.

Тогда, показания акселерометра $w^{\textit{gyr}}$ складываются из угловой скорости сенсора, (может быть еще чего-то) и ошибки измерений $e^{\textit{gyr}}$.

\begin{align} 
&q_{BG} \circ (w^{\textit{gyr}} - e^{\textit{gyr}}) \circ \tilde{q}_{BG} = 
\Omega_B
\\
\end{align}

Будем считать, что ошибка измерений гироскопа складывается из постоянной компоненты  и белого шума
\begin{align} 
e^{\textit{gyr}} = e_{0}^{\textit{gyr}} + \epsilon^{\textit{gyr}}
\end{align}

Смещение и поворот собственных осей, а также постоянную компоненту ошибки измерений $e_{0}^{\textit{gyr}}$ можно найти с помощью калибровки.

\subsection{Система глобальной навигации}
Обозначим $\delta r^{gnns}$ смещение принимающей антенны систем глобальной навигации. Тогда измерения положения и скорости связаны с положением начала собственной системы координат как
\begin{align} \label{eq:gnns_mes_model}
\begin{split}
&r_{gnns} = r_I + q_{IB} \circ \delta r^{gnns}_B \circ \tilde{q}_{IB} \\
&v_{gnns} = v_I + q_{IB} \circ  (\Omega_B \times \delta r^{gnns}_B) \circ \tilde{q}_{IB}
\end{split}
\end{align}

\section{Оценка состояния}
Для оценки состояния используем алгоритм SR-EKF (квадратно-корневой расширенный фильтр Калмана). Вектор состояния составим из положения, скорости, ускорения, кватерниона ориентации и угловой скорости
\begin{align} 
&x = (r_I \quad \dot{r}_I \quad \ddot{r}_I \quad q_{BI} \quad \Omega_B)^T
\end{align}
!!!Тут стоит описать алгоритм!!!

\subsection{Арпиори оценка}
Априори оценка ускорения и угловой скорости производится на основе показаний датчиков ускорения и угловой скорости. Остальные компоненты оцениваются при помощи интегрирования.
Зная калибровочные параметры датчика и делая некоторые упрощения для уравнения \eqref{eq:imu_mes}, будем считать, что
\begin{align} 
\begin{split}
&\ddot{r}_{I} = {q}_{BI} \circ  [a^{imu}_{calib} + (\Omega_B \times (\delta r^{imu}_B \times \Omega_B))] \circ \tilde{q}_{BI} + g_I
\end{split}
\end{align}
Здесь $a^{imu}_{calib}$ -- скорректированные на поворот и постоянную компоненту ошибки показания акселерометра
\begin{align} 
\begin{split}
a^{imu}_{calib} = q_{BM} \circ (a^{imu} - e^{imu}_0) \circ \tilde{q}_{BM}
\end{split}
\end{align}
Для угловой скорости запишем
\begin{align} 
\Omega_B = w^{\textit{gyr}}_{calib}
\end{align}
\begin{align} 
&w^{\textit{gyr}}_{calib} = q_{BG} \circ (w^{\textit{gyr}} - e_0^{\textit{gyr}}) \circ \tilde{q}_{BG}
\end{align}
С учетом этих уравнений и уравнений модели  (\ref{eq:model_velocity}-\ref{eq:model_rot_velocity})
\begin{align}
\begin{split}
&F = \frac{f(x,t)}{\delta x} = \\
&\begin{pmatrix}
O_{3x3} & E_{3x3} & O_{3x3} & O_{3x4} & O_{3x3}\\
O_{3x3} & O_{3x3} & E_{3x3} & M^{\ddot{r}}_{\delta q} &M^{\ddot{r}}_{\delta \Omega}\\
O_{3x3} & O_{3x3} & O_{3x3} & O_{3x4} & O_{3x3}\\
O_{4x3} & O_{4x3} & O_{4x3} & M^{\dot{q}}_{\delta q} & M^{\dot{q}}_{\delta \Omega}\\
O_{3x3} & O_{3x3} & O_{3x3} & O_{3x4} & O_{3x3}
\end{pmatrix}
\end{split}
\end{align}
!!!Здесь $M$ -- матрицы, которые нужно описать здесь.!!!

\subsection{Коррекция GNNS положения и скорости}
Для коррекции используются показания систем глобальной навигации, которые позволяют оценить положение и скорость антенны
\begin{align}
z_{gnns} = (r_{gnns} \quad v_{gnns})^T.
\end{align}
Используя уравнения \eqref{eq:gnns_mes_model} можем записать 
\begin{align}
\begin{split}
&H_{gnns} = \frac{h_{gnns}}{\delta z_{gnns}} =  \\
&\begin{pmatrix}
E_{3x3} & O_{3x3} & O_{3x3} & Z^r_q & O_{3x3}   \\
O_{3x3} & E_{3x3} & O_{3x3} & Z^v_q & Z^v_{\Omega}
\end{pmatrix}
\end{split}
\end{align}
!!!Здесь $Z$ -- матрицы, которые нужно описать здесь.!!!

\subsection{Коррекция GNNS абсолютной величины и направления скорости}
Используя уравнения модели (\ref{eq:model_velocity}-\ref{eq:model_rot_velocity}) и уравнения для показаний GNNS датчика \eqref{eq:gnns_mes_model} можно записать
\begin{align}
\begin{split}
&v_{gnns} = v_I + q_{IB} \circ  (\Omega_B \times \delta r^{gnns}_B) \circ \tilde{q}_{IB} = \\
& = q_{IB} \circ (|v_I| \quad 0 \quad 0)^T \circ \tilde{q}_{IB} + q_{IB} \circ  (\Omega_B \times \delta r^{gnns}_B) \circ \tilde{q}_{IB}
\end{split}
\end{align}
!!!отсюда можно сделать второй вариант коррекции!!!
\end{document}