\documentclass[a4paper,12pt]{article}

\usepackage{cmap}
\usepackage{amsmath}
\usepackage[T2A]{fontenc}
\usepackage[utf8]{inputenc}
\usepackage[english,russian]{babel}


\begin{document}
\section{Модель движения}

Положение робота описывается в некоторой локальной системе отсчета $I$, ось $Z_I$ направлена против гравитации, оси $X_I$ и $Y_I$ дополняют систему до правой тройки.

Ориентация робота представлена кватернионом $q$ таким образом, что произвольный вектор, записанный в собственных осях $B$ робота проецируется в локальную систему как
$r_I = q_{IB} \circ r_B \circ \tilde{q}_{IB}$

Движение робота определено текущей ориентацией и управляющим воздействием на приводы колес $v$.
\begin{align*} 
&\dot{r}_B = (v \quad 0 \quad 0)^T \\
&\dot{r}_I = q_{IB} \circ \dot{r}_B \circ \tilde{q}_{IB}
\end{align*}

Текущая ориентация робота определяется поворотом руля, которая определяет локальную кривизну траектории $u$, и вектором нормали $n_s$ к поверхности, по которой перемещается робот.
\begin{align*} 
&q_{IB} = q_{\phi} \circ q_{s} \\
&q_{\phi} = (\cos{\frac{\phi}{2}} \quad n_{s}\sin{\frac{\phi}{2}})^T \\
&\dot{\phi} = vu \\
&q_{s} = (\cos{\frac{\psi}{2}} \quad n_{\perp}\sin{\frac{\psi}{2}})^T \\
&\cos{\psi} = e_z \cdot n_s \\
& n_{\perp} ||n_{\perp}|| = e_z \times n_s 
\end{align*}

Угловая скорость связана с кватернионом ориентации уравнением Пуассона
\begin{align*} 
&\dot{q}_{IB} = \frac{1}{2} q_{IB} \circ \Omega_{B} \\
\end{align*}
Поверхность может быть задана как $f_s(x,y,z) = 0$,
тогда 
$$n_s = \frac{\nabla f_s}{||\nabla f_s||}.$$


\section{Модель измерений}
	
\subsection{Датчик линейного ускорения}
Обозначим $\delta r^{imu}$ смещение акселерометра,  $q_{BM}$ -- ориентацию собственных осей акселерометра относительно осей $B$, тогда для произвольного вектора
$r_B = q_{BM} \circ r_M \circ \tilde{q}_{BM}.$

Тогда, показания акселерометра $a^{imu}$ складываются из переносной, вращательной и центробежной компонент и ускорения свободного падения и ошибки измерений $e^{imu}$.

\begin{align*} 
&q_{BM} \circ (a^{imu} - e^{imu}) \circ \tilde{q}_{BM} = 
{q}_{BI} \circ (-\ddot{r}_{I} - g_I) \circ \tilde{q}_{BI}
 - {\dot{\Omega}_B \times \delta r^{imu}_B} - \Omega_B \times (\delta r^{imu}_B \times \Omega_B)
\\
&q_{BI} = \tilde{q}_{IB}
\end{align*}

Ошибка измерений сосоит из следующих компонент: шум и биас. Нужно пострить их модели.

Смещение и поворот собственных осей можно найти с помощью калибровки.

\subsection{Датчик угловой скорости}
Обозначим $\delta r^{\textit{gyr}}$ смещение гироскопа,  $q_{BG}$ -- ориентацию собственных осей гироскопа относительно осей $B$.

Тогда, показания акселерометра $w^{\textit{gyr}}$ складываются из угловой скорости сенсора , (может быть еще чего-то) и ошибки измерений $e^{\textit{gyr}}$.

\begin{align*} 
&q_{BG} \circ (w^{\textit{gyr}} - e^{\textit{gyr}}) \circ \tilde{q}_{BG} = 
\Omega_B
\\
\end{align*}

Ошибка измерений сосоит из следующих компонент: шум и биас. Нужно пострить их модели.

Смещение и поворот собственных осей можно найти с помощью калибровки.



\end{document}